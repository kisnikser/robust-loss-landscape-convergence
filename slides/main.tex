\documentclass[aspectratio=169]{beamer}

% \graphicspath{{figs}}  % your figures here

\usepackage[T1,T2A]{fontenc}
\usepackage[main=russian,english]{babel}

\usepackage{cmap}
\usepackage{tikz}
\usepackage{array}
\usepackage{multicol}
\usepackage{booktabs}
\usepackage{csquotes}
\usepackage{amssymb,amsfonts,amsmath}

\usecolortheme{beaver}
\usefonttheme[onlymath]{serif}  % math fonts as in article
\setbeamertemplate{navigation symbols}{}
\setbeamertemplate{footline}[page number]
\setbeamertemplate{blocks}[rounded=true,shadow=true]
\setbeamersize{text margin left=0.5cm,text margin right=0.5cm}
\setbeamerfont{author}{size=\normalsize}
\setbeamerfont{institute}{size=\normalsize}
\setbeamerfont{date}{size=\normalsize}
\setbeamertemplate{enumerate item}{\insertenumlabel)}
\setbeamertemplate{enumerate subitem}{\insertenumlabel)}
\setbeamertemplate{enumerate subsubitem}{\insertenumlabel)}

\title{Название}
\author{
    Фамилия~И.\,О.\\
    Научный руководитель: ученая степень, ученое звание Фамилия~И.\,О.
}
\institute{
    Кафедра интеллектуальных систем ФПМИ МФТИ\\
    Специализация: Специализация из ЛК МФТИ\\
    Направление: Направление из ЛК МФТИ
}
\date{Дата}

\begin{document}

\begin{frame}
    \thispagestyle{empty}
    \maketitle
\end{frame}

\begin{frame}{Название или его основная часть}
    Пара предложений с мотивацией ...
    \begin{block}{Проблема}
        Общие слова о проблеме ... 
    \end{block}
    \begin{block}{Требуется}
        Основная идея ...
    \end{block}
    \begin{block}{Метод решения}
        Предлагаемый метод ...
    \end{block}
\end{frame}

\begin{frame}{Один слайд, представляющий всю работу}
    Что аудитория должна увидеть на изображении.
    \begin{columns}
    \begin{column}{0.5\textwidth}
        \begin{enumerate}
            \item Обозначения
            \item Формулы
        \end{enumerate}
    \end{column}
        \begin{column}{0.5\textwidth}
            \begin{figure}[h]
                \centering
                \includegraphics[width=0.8\textwidth]{example-image}
                \caption{Название}
            \end{figure}    
        \end{column}
    \end{columns}
    Какие выводы можно сделать?
\end{frame}

\begin{frame}{Постановка задачи}
    Задан набор данных $\mathfrak{D} = \left\{ (\mathbf{x}_i, y_i) \right\}_{i=1}^{N}$ ...
\end{frame}

\begin{frame}{Метод}
    Предлагается ...
\end{frame}

\begin{frame}{Вычислительный эксперимент}
    Рассматривается набор данных MNIST ...
    \begin{columns}
        \begin{column}{0.5\textwidth}
            \begin{figure}[h]
                \centering
                \includegraphics[width=0.7\textwidth]{example-image-a}
                \caption{Название A}
            \end{figure}    
        \end{column}
        \begin{column}{0.5\textwidth}
            \begin{figure}[h]
                \centering
                \includegraphics[width=0.7\textwidth]{example-image-b}
                \caption{Название B}
            \end{figure}    
        \end{column}
    \end{columns}
    Результаты показывают, что ...
\end{frame}

\begin{frame}{Выносится на защиту}
    \begin{enumerate}
        \item Рассматривается проблема ...
        \item Предлагается ...
        \item Результаты вычислительного эксперимента показывают, что ...
        \item Предлагаемый метод превосходит ...
    \end{enumerate}
\end{frame}

\begin{frame}{Список работ автора по теме диссертации}
    \small
    {\usebeamercolor[fg]{block title} Публикации в журналах из списка ВАК}\\
    \begin{enumerate}
        \item Публикация 1
        \item Публикация 2
    \end{enumerate}
    {\usebeamercolor[fg]{block title} Публикации по итогам конференций, индексируемые в международных базах данных}\\
    \begin{enumerate}
        \item Публикация 1
        \item Публикация 2
    \end{enumerate}
    {\usebeamercolor[fg]{block title} Выступления с докладом}\\
    \begin{enumerate}
        \item Выступление 1
        \item Выступление 2
    \end{enumerate}
\end{frame}

\end{document} 